\documentclass[11pt, reqno]{amsart}

\edef\restoreparindent{\parindent=\the\parindent\relax}
\usepackage{parskip}
\restoreparindent
\usepackage{bbm}

\usepackage{xcolor}
\usepackage{multicol}

\usepackage{enumitem}
\usepackage{amsmath, amssymb, mathrsfs}
\usepackage{graphicx}
\usepackage{url}

\usepackage{etoolbox}
\patchcmd{\section}{\scshape}{\Large\scshape}{}{}

\usepackage[width=6.5in,height=9.5in]{geometry}


\usepackage[bookmarks,colorlinks,breaklinks]{hyperref} 
\usepackage{color}
\definecolor{dullmagenta}{rgb}{0.4,0,0.4}   % #660066
\definecolor{darkblue}{rgb}{0,0,0.4}
\hypersetup{linkcolor=darkred,citecolor=blue,filecolor=dullmagenta,urlcolor=darkblue} % coloured links

\setlength{\skip\footins}{2cm}

\begin{document}
	\title{\LARGE CS6390 Project Proposal: REDIRE/GIRAF}
	\author{\textsc Oliver Richardson \hspace{2em} Maks Cegielski-Johnson }
	\maketitle 
	
	\vspace{-2em}
	
	\section{Task}

	Our interest in this problem is two-fold; we are interested in studying both recognition and generation of paraphrases. Hence we will need to discuss our current approaches to both of these problems. 
4	
	The paraphrase recognition task is defined as attempting to determine whether two sentences are paraphrases of each other, returning a \textsc{yes/no} answer.
	
	The paraphrase generation task is defined as being given an input sentence, and trying to generate a paraphrase from that input. \textcolor{red}{Deciding correctness is not defined yet.}
	

	
	\section{Architecture}
	Our current implementation is inspired by a paper by Malakasiotis\cite{malakasiotis2009paraphrase}. The general structure of this implementation is as follows:
	
	\begin{enumerate}
	\item Restructure each pair of sentences $s_1$ and $s_2$ into 10 new string transformations $t_{i,1}$ and $t_{i,2}$, maintaining the original order
		\begin{multicols}{2}
		\begin{enumerate}
			\item String of the tokens
			\item String of the stems.
			\item String of the POS tags.
			\item String of the soundex codes\cite{soundex}.
			\item String of all the noun tokens
			\item String of all the noun stems
			\item String of all the noun soundex codes.
			\item String of all the verb tokens.
			\item String of all the verb stems.
			\item String of all the verb soundex codes.
		\end{enumerate}
		\end{multicols}
	\item For each string pair $s_1$ and $s_2$, and for each transformation of that string $t_{i,1}$ and $t_{i,2}$, we want to compute the similarity $d_i(t_{i,1},t_{i,2})$, using the metrics 
		\begin{multicols}{2}
		\begin{enumerate}
		\item Levenshtein (word edit)
		\item Jaro-Winkler
		\item Manhattan \& Euclidean
		\item Cosine
		\item $n$-gram ($n$ = 3)
		\item Overlap
		\item Dice coefficient
		\item Jaccard coefficient 
		\end{enumerate}
		\end{multicols}
	\item Computing the similarities between each transformation of the two strings results in 90 features. 
	\item Next, for each pair of sentences $s_1$ and $s_2$, if $s_1$ is longer than $s_2$ we compute the substrings of $s_1$ which have the same length as $s_2$. For each substring $s_1'$ generated, we compute all of the similarities between $s_1'$ and $s_2$. We try to find which substring results in the max average similarity over all the similarity metrics, namely $s_1'^*$. Then for $s_1'^*$ we compute each of the similarities between it and $s_2$, as well as the average of all of them, resulting in 10 features. We do this for the string transformations which result in tokens, stems, POS tags, and soundex codes. This results in an additional 40 features per sentence pair (total of 130 features so far). Note that we can just swap the two strings in the discussion above if $s_2$ is larger than $s_1$. 
	
	\item Finally, we add three additional features to the current 130. We add a boolean feature for whether $s_1$ contains a negation, and a feature for whether $s_2$ contains negation. Then finally, we add a feature for the length ratio, defined as $\frac{\min(|s_1|, |s_2|)}{\max(|s_1|, |s_2|)}$.
	\end{enumerate}
	
	\section{Experiments}
	
	\section{Results}
	

	
	\section{Contributions}
	
	\bibliographystyle{plain}
	\bibliography{references}
		
\end{document}